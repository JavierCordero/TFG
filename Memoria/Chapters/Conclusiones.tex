\parindent=0em
\chapter{Conclusiones y trabajo futuro}
\noindent

Después de la investigación de las distintas realidades abarcadas en las XR, centrándonos en la realidad mixta y realizando distintas pruebas sobre este entorno se han obtenido los siguientes resultados. \\

La realidad mixta a día de hoy es una tecnología poco madura y que necesita un desarrollo más prolongado. Esta cuenta con una gran variedad de dispositivos, aunque las grandes industrias se centran más en el desarrollo de equipos para realidad virtual que en realidad aumentada o mixta. Esto hace que los costes de estas últimas sean más elevados y se encuentren menos en el mercado. Además, el desarrollo en HMDs de realidad mixta es mucho mayor en comparación a su desarrollo en teléfonos móviles.\\

Sin embargo, la tecnología de realidad aumentada se ve destinada a su uso principal en dispositivos móviles. Gracias a tecnologías como ARCore que innova en el apartado de generación de nube de puntos y permite que esta se genere exclusivamente con una cámara, facilita el acceso a esta tecnología a un mayor número de personas.\\

En el entorno del videojuego, este esta destinado a ser principalmente generado para dispositivos móviles. A día de hoy no existen demasiadas referencias dentro de este ámbito y las pocas que se pueden encontrar no son de realidad mixta, sino que su foco principal se centra en la realidad aumentada.\\

Gracias a las pruebas que se han realizado con la nube de puntos, se puede afirmar que esta novedosa tecnología de cálculo de profundidad es insuficiente para detección de planos o de superficies que no contengan bordes. También se ha llegado a la conclusión de que la nube de puntos funciona correctamente. Aun así, debido a las características de los dispositivos móviles actuales, su rendimiento queda bastante por debajo del esperado para una aplicación estable.\\

Por otro lado, después de documentarse sobre distintos servicios de búsqueda de rutas, podemos afirmar que Mapbox es la herramienta que mejor se adapta. Teniendo en cuenta que presta servicios sin necesidad de invertir dinero, de una forma sencilla y accesible en teléfonos móviles. A través de Mapbox se puede obtener la información de las rutas de forma muy detallada. Además, la extensa documentación de esta tecnología facilita el desarrollo de aplicaciones.\\

Como trabajo futuro a realizar, en primer lugar se debería realizar una investigación exhaustiva sobre los dispositivos móviles actuales, sus características a nivel hardware (CPU, memoria principal, procesamiento de imágenes...) y en base a ello realizar una predicción sobre cómo serán estos en el futuro.\\ 

Dicha predicción serviría para determinar en qué momento la tecnología de realidad mixta puede llegar a ser viable en dispositivos móviles y ejecutada en tiempo real en estos.\\

No se debe olvidar tampoco la existencia de los sensores \textit{ToF}, aunque la integración de estos en los dispositivos móviles encarece bastante su producción y venta, por lo que no lo consideramos óptimo para su comercialización.\\

Mientras se alcanza este punto tecnológico, se debería optimizar al máximo la forma en la que se generan nubes de puntos. Para ello, se debería añadir la detección por superficies utilizando dicha tecnología y no solo como una funcionalidad más de la aplicación.\\

A su vez, también se debería investigar distintas formas de generar mallas 3D en base a una nube de puntos, las cuales presenten coherencia con esta y se apliquen a las superficies de los objetos.\\ 

Respecto a los sistemas de búsqueda de rutas, se podrían realizar las mismas pruebas que en el apartado~\ref{pruebas geo arcore} pero en este caso, utilizando la tecnología A-GPS que se ha tratado en el mismo apartado. De esta forma se podría comprobar la exactitud de la ubicación del usuario, lo cual permitiría generar un entorno de realidad mixta más preciso. 