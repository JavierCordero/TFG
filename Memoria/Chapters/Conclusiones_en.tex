\parindent=0em
\addtocounter{chapter}{-1}
\chapter{Conclusions}
\noindent

After researching the different realities covered in the XR, focusing on mixed reality and conducting different tests on this environment, the following results have been obtained. \\

Today's mixed reality is a technology that is not very mature and needs a longer development. It has a wide variety of devices, although large industries focus more on the development of equipment for virtual reality than on augmented or mixed reality. This makes the costs of the XR devices higher and there are less on the market. Furthermore, the development in mixed reality HMDs is much greater compared to its development in mobile phones. \\

However, augmented reality technology is intended for primary use on mobile devices. Thanks to technologies such as ARCore that innovates in the point cloud generation section and allows it to be generated exclusively with a camera, it facilitates access to this technology for a greater number of people. \\

In the video game environment, this is intended to be mainly generated for mobile devices. Today there are not too many references in this area and the few that can be found are not of mixed reality, but their main focus is on augmented reality. \\

Thanks to the tests that have been carried out with the point cloud, it can be stated that this new depth calculation technology is insufficient for detecting planes or surfaces that do not contain edges. It has also been concluded that the point cloud works correctly. Still, due to the characteristics of current mobile devices, its performance is well below what is expected for a stable application. \\

On the other hand, after documenting about different route search services, we can affirm that Mapbox is the tool that best adapts. Taking into account that it provides services without the need to invest money, in a simple and accessible way on mobile phones. Through Mapbox you can obtain the information of the routes in a very detailed way. In addition, the extensive documentation of this technology facilitates the development of applications. \\

As future work to be carried out, in the first place, an exhaustive investigation should be carried out on current mobile devices, their characteristics at the hardware level (CPU, main memory, image processing ...) and based on this, make a prediction on how they will be these in the future. \\

This prediction would serve to determine at what moment the mixed reality technology can become viable on mobile devices and run in real time on them. \\

The existence of \textit{ToF} sensors should not be forgotten either, although the integration of these in mobile devices makes their production and sale more expensive, so we do not consider it optimal for their commercialization. \\

While this technological point is reached, the way in which point clouds are generated should be maximized. For this, surface detection should be added using this technology and not just as one more functionality of the application. \\

In turn, it could be investigated different ways to generate 3D meshes based on a point cloud, which are consistent with it and applied to the surfaces of objects. \\

Regarding to route search systems, the same tests could be carried out as in section ~\ref{geo arcore tests}, but in this case, using the A-GPS technology that has been discussed in the same section. In this way, the accuracy of the user's location could be verified, which would allow a more accurate mixed reality environment to be generated.