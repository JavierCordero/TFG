\parindent=0em
\section{Dispositivos}
\noindent

%https://editeca.com/realidad-mixta/

Actualmente los dispositivos más utilizados para la realidad mixta son los cascos también conocidos como diademas, los cuales, vienen incorporados con: cámaras, sensores infrarrojos, micrófonos y acelerómetros entre otros. \\\\Estos dispositivos se pueden clasificar en dos grupos:

\begin{itemize}
    \item \textbf{Dispositivos holográficos:} Se trata de aquellos que crean imágenes tridimensionales basadas en el empleo de la luz (hologramas) haciendo que el contenido parezca que está realmente en el mundo físico.
    
    \item \textbf{Dispositivos envolventes:} Estos dispositivos son capaces de remplazar el mundo físico por una experiencia digital, bloqueando el entorno físico.
    
\end{itemize}

Por otra parte, los teléfonos móviles están empezando a ganar importancia en el sector de la realidad mixta. Según un estudio realizado por Sam Barker \cite{juniperArMrmoney} , la aparición de las redes de  5G y los avances del Edge Computing acelerarán el desarrollo de la tecnología de realidad mixta en móviles llegando a alcanzar un mercado de 43 billones de dólares en el año 2024, a diferencia de los 8 billones alcanzados en 2019.

\parindent=0em
\subsection{Hololens 2}
\label{HoloLens2Dispositivo}
\noindent

Las \textit{Microsoft HoloLens 2} son un casco desarrollado por Microsoft que salió al mercado por primera vez el día 7 de Noviembre de 2019, es el dispositivo sucesor a las Microsoft Hololens de 2016.


\begin{figure}[htbp]
\centering
    \hspace{-4mm}
    \begin{minipage}{0.5\textwidth}
        \centering
        \includegraphics[scale=0.2]{Images/Estado del arte/hololens2_2.jpeg}\\
        (a) Vista lateral.
    \end{minipage}
    \begin{minipage}{0.5\textwidth}
        \centering
        \includegraphics[scale=0.2]{Images/Estado del arte/hololens2_3.jpeg}\\
       (b) Vista superior.
    \end{minipage}\\
    \caption{Vistas de las Microsoft HoloLens 2\textsuperscript{\ref{hololens2footer}}.}
    \label{fig:vistasHoloLens2}
\end{figure}


Microsoft Hololens 2 utiliza los servicios de cloud y la IA de Microsoft, además, está diseñado con un sistema ergonómico de tal forma que no hay necesidad de quitarse las gafas ya que se coloca directamente sobre ellas (figura~\ref{fig:hololensErgonomicas}).\\
\begin{wrapfigure}{r}{.5\textwidth}
  \centering
    \includegraphics[scale=0.28]{Images/Estado del arte/hololens2_1.jpeg}
  \caption{Diseño ergonómico del dispositivo\textsuperscript{\ref{hololens2footer}}
  }
  
  \label{fig:hololensErgonomicas}
\end{wrapfigure}

Respecto a las especificaciones técnicas del casco se pueden destacar las siguientes relacionadas con los sensores: cuatro cámaras de luz visible para el seguimiento de la cabeza, dos cámaras de infrarrojos para el movimiento de los ojos y un sensor de profundidad de 1 mega píxel, además de contar con acelerómetro, giroscopio y magnetómetro.\\



El dispositivo cuenta también con seguimiento de mano, seguimiento de ojos y controles por voz.
Por último, las Microsoft HoloLens 2 tienen un peso de 566 gramos y utilizan un sistema operativo holográfico de Windows.

Actualmente están disponibles para empresas y desarrolladores, su precio depende de la elección entre distintos paquetes:

\begin{itemize}
    \item \textbf{Solo el dispositivo:} 3,500 dólares.
    \item \textbf{Dispositivo con asistencia remota:} El coste del dispositivo y 500 dólares al mes, incluye asistencia remota los 365 días.
    \item \textbf{Paquete de desarrollador:} Coste del dispositivo añadido a 99 dólares por mes, este paquete incluye Unity Pro y un crédito de 500 dólares para Azure.
\end{itemize}

\footnotetext[1]{\label{hololens2footer}{Imágenes y especificaciones obtenidas de \url{https://www.microsoft.com/es-es/hololens/hardware}.}}




\parindent=0em
\subsection{HMD Odyssey+}
\label{sec:odyssey}
\noindent

%https://www.samsung.com/hk_en/news/product/reality-headset-hmd-odyssey-plus/

El casco \textit{HMD Odyssey+} (figura~\ref{fig:hdmOdysseyVista}) es un dispositivo desarrollado por la empresa Samsung que salió al mercado en 2018, no es un casco independiente ya que requiere estar conectado a un ordenador compatible con la plataforma \textit{Windows Mixed Reality} para funcionar, se conecta al ordenador mediante un cable HDMI~2.0 y un USB~3.0.

\begin{figure}[h]
    \centering
    \includegraphics[scale=0.6]{Images/Estado del arte/samsungOdysseyplus.jpg}
    \caption[\textit{HMD Odyssey+ dispositivo al completo}]{\textit{HMD Odyssey+ dispositivo al completo}\footnotemark.}
    \label{fig:hdmOdysseyVista}
\end{figure}

\footnotetext{Fuente: \url{https://www.samsung.com/hk_en/news/product/reality-headset-hmd-odyssey-plus/}}
En cuanto a la parte de monitorización de la imagen, este casco tiene una resolución de 1.440 x 1.600 píxeles por ojo, un total de 2.880 x 1.600 píxeles combinando los dos ojos, además, el \textit{Field of View} abarca un total de 110\degree . Este dispositivo no posee tecnología de \textit{hand tracking} (por lo que controla el movimiento de manos con dos mandos que funcionan con pilas) y tampoco tiene \textit{eye tracking}, en cambio, destaca por tener dos sensores 6DoF frente al único que tiene, por ejemplo, las \textit{Hololens~2} (sección~\ref{HoloLens2Dispositivo}).\\

Por otra parte, está dotado de sensores como acelerómetro, giroscopio, brújula y sensor de proximidad (este último se utiliza para saber en qué momento se lleva puesto el casco), del mismo modo, posee una diadema que se puede  modificar para ajustar el casco a la cabeza de cada individuo, sin embargo, este dispositivo no está diseñado para ser utilizado con gafas.\\ 

Por último, tiene un sensor que ajusta automáticamente la IPD, conexión \textit{bluetooth} y un peso total de 644g teniendo en cuenta solo el HMD y un añadido de 176g si se cuenta el peso del cable.

\parindent=0em
\subsection{Magic Leap One}
\noindent

El dispositivo \textit{Magic Leap One} (figura~\ref{magicLeaponeFotterSpecs}) fue lanzado al mercado en el año 2018 de manos de la compañía Magic Leap. Principalmente este casco destaca por su ligero peso frente a otros dispositivos de 316 gramos, por otra parte, son un \textit{HMD} independiente ya que vienen con un sistema (procesador y tarjeta gráfica) para no tener que ser usados mediante conexión con un ordenador. Aunque no es necesario conectarlo al ordenador, el dispositivo tiene conexión \textit{bluetooth}, \textit{WiFi} y \textit{USB} para poder conectarse al \textit{PC} y manejar archivos o instalar aplicaciones.\\

Por otra parte, pese a que dispone de tecnología de \textit{hand tracking}, viene con un mando para simular una de las manos del usuario, asimismo, utiliza la tecnología de \textit{eye tracking} para poder interactuar con el entorno a través de la mirada. Del mismo modo, el dispositivo captura \textit{6DoF}, tiene un campo de visión de 50º y una resolución de 1.280x960 píxeles por ojo (un total de (2.560x960 píxeles en combinación con los dos ojos).
 

\begin{figure}[htbp]
\centering
    \hspace{-4mm}
    \begin{minipage}{0.5\textwidth}
        \centering
        \includegraphics[scale=0.8]{Images/Estado del arte/magicleapone1.jpg}\\
        (a) Vista del casco completo
    \end{minipage}
    \begin{minipage}{0.5\textwidth}
        \centering
        \includegraphics[scale=0.3]{Images/Estado del arte/magicleapone2.jpg}\\
       (b) Vista del mando
    \end{minipage}\\
    \caption{Dispositivo completo de las Magic Leap One\textsuperscript{\ref{magicLeaponeFotterImages}}.}
    \label{fig:vistasMagicLeapnOne}
\end{figure}

Por último, la batería recargable de litio de las \textit{Magic Leap One} permite un uso continuado de 3 horas y, además, tiene un diseño ergonómico mediante el cual se puede utilizar el casco con gafas, ya que hay un compartimento enfrente de los ojos adaptado para el hueco de unas gafas.

%https://www.businessinsider.com/magic-leap-one-creator-edition-price-specifications-battery-life-release-date-2018-8?IR=T


\footnotetext[1]{
\label{magicLeaponeFotterSpecs}{Especificaciones obtenidas de \url{https://www.businessinsider.com/magic-leap-one-creator-edition-price-specifications-battery-life-release-date-2018-8?IR=T}.}}

\footnotetext[1]{
\label{magicLeaponeFotterImages}{Imágenes sacadas de \url{https://www.estiloextra.net/magic-leap-one-las-nuevas-y-prometedoras-gafas-de-realidad-aumentada/}.}}

\parindent=0em
\subsection{DAQRI Smart Helmet}
\noindent

%https://www.linkedin.com/pulse/daqri-smart-helmet-closer-look-nathan-gaydhani

El \textit{DAQRI Smart Helmet} es un \textit{HMD} enfocado al uso industrial, está formado por un dispositivo de realidad mixta sin cables integrado en un casco duro (cascos utilizados por ejemplo en la construcción). Ya que está pensado para un uso industrial está dotado de 4 cámaras para poder capturar todo el entorno, es decir, abarcar 360º.\\

Con el objetivo de proteger al usuario en entornos industriales peligrosos, el dispositivo cuenta con una cámara termográfica (para poder controlar lugares potencialmente peligrosos debido a su temperatura) y barómetro para poder medir la presión del entorno.\\

Para el seguimiento, el \textit{DAQRI Smart Helmet} utiliza la tecnología \textit{SLAM} y posee \textit{6DoF} para el movimiento y los giros del usuario. Del mismo modo, ya que el objetivo final es proteger al usuario, el dispositivo cuenta con reconocimiento de órdenes por voz y de control de movimientos de la cabeza para manejar el \textit{HMD}, de esta forma se pueden tener las manos libres.\\

\begin{figure}[h]
    \centering
    \includegraphics[scale=0.12]{Images/Estado del arte/daqrihelmet.jpg}
    \caption{Vista del DAQRI Smart Helmet}
    \label{fig:vistaDAQRIHelmet}
\end{figure}

Finalmente, cabe recalcar que el casco goza de procesadores y un sistema operativo \textit{Android} para ser totalmente independiente, también, utiliza unas baterías intercambiables de 5700 miliamperios por hora y dispone de conectividad \textit{WiFi} para poder comunicarse por vídeo en tiempo real remotamente con otros usuarios. El casco tiene un peso total de 1 kilo y~500 gramos y su precio actual es de~15.000 dólares.

%\footnotetext{\label{daqriImagefooter}{Imagenes obtenida de: %\url{https://www.stereoscape.com/blog/2017/04/25/daqri-smart-helmet-so-m%uch-more-than-a-helmet/}.}}


