\parindent=0em
\subsection{Mapbox}
\noindent

Mapbox~\cite{mapboxMainPage} es una herramienta de código abierto de localización para móviles y entornos web. Mediante Mapbox se pueden crear aplicaciones con funciones como mapas, navegación o búsquedas.\\

Las distintas funciones que ofrece Mapbox son:

\begin{itemize}
    \item \textit{\textbf{Maps}}: API que ofrece la visualización de mapas dinámicos y personalizables.
    
    \item \textit{\textbf{Navigation}}: Motor de búsqueda de rutas preciso con actualizaciones en tiempo real de accidentes, dependiente de si el usuario va caminando, en bicicleta o en coche. Esta característica de Mapbox permite la navegación por voz y la navegación sin conexión a internet (a tavés de mapas ya descargados).

    \item \textit{\textbf{Atlas}}: \textit{Atlas} es una plataforma de desarrollo de aplicaciones de Mapbox en la nube, permite al usuario guardar los datos de su aplicación en la nube (para poder trabajar en ella desde cualquier lado).
    
    \item \textit{\textbf{Search}}: Esta herramienta ofrece la posibilidad de dar un contexto al usuario final de la aplicación cambiando sus coordenadas GPS por el nombre de su ciudad o localización, además, permite colocar etiquetas con nombres en el mapa.
    
    \item \textit{\textbf{Studio}}: Herramienta web que permite crear un estilo propio de mapas.
    
    \item \textit{\textbf{Vision}}: Esta herramienta utiliza la cámara del móvil para funcionar como un copiloto al conducir un coche controlado por inteligencia artificial, este SDK aporta funciones como alertar al conductor cuando no cumple la distancia de seguridad o cuando sobrepasa la velocidad permitida.
    
    \item \textit{\textbf{Data}}: Permite obtener un gran conjunto de datos para analizar desde información sobre las fronteras hasta movimiento de coches y tráfico.
\end{itemize}

Aunque Mapbox es una herramienta de código abierto, los servicios mencionados anteriormente funcionan bajo demanda. Estas funciones tienen un plan gratuito limitado que se puede ampliar pagando. Por ejemplo, se puede aumentar el número de usuarios activos del servicio \textit{Maps} de los 25.000 del plan gratuito a 125.000 comprando el primer plan de pago.