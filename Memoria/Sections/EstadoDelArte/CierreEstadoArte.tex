\parindent=0em
\section*{Resumen}
\noindent

En este capítulo se han tratado los conceptos de las tres realidades que abarcan las XR. Dichos conceptos han servido para poder distinguir las distintas realidades entre ellas, además, se han contado las tecnologías que hacen falta para sus usos. Dentro de estas tecnologías cabe destacar la importancia del \textit{tracking} a través de distintas técnicas.\\

Una vez conocidos lo componentes necesarios para su funcionamiento se puede hablar de los distintos dispositivos que permiten su uso, desde teléfonos móviles hasta HMDs donde se ha podido ver que estos tienen distintos tipos de conectividad, pueden utilizar mandos o no e incluso estar dotados de tecnología \textit{eye tracking}.

Por último, la realidad mixta potencia distintos campos como la educación, la atención médica, la industria o los videojuegos. Para poder desarrollar aplicaciones de realidad aumentada y realidad mixta destacan distintas APIs como ARCore, ARKit o Vuforia.\\

En este capítulo se ha visto el funcionamiento de las XR y sus usos. En el siguiente capítulo se podrán ver distintas aplicaciones haciendo uso de la API ARcore para conseguir un efecto de oclusión.