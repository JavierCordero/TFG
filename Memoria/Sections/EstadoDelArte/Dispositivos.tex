\parindent=0em
\section{Dispositivos}
\noindent

%https://editeca.com/realidad-mixta/
Dado que la realidad aumentada y la realidad virtual tienen distintas necesidades tecnológicas como se ha tratado en la sección bla bla bla , sus dispositivos correspondientes también tienen componentes hardware distintos.\\

En el sector de la realidad aumentada destaca principalmente el uso de teléfonos móviles gracias a su fácil accesibilidad y a la facilidad añadida de crear aplicaciones para estos gracias a la aparición de las tecnologías \textit{ARCore} y \textit{ARKit}, además, se puede disfrutar de esta experiencia usando gafas de realidad aumentada y \textit{HMDs}.\\

Por otra parte, en el campo de la realidad virtual destacan los \textit{HMD} también conocidos como cascos de realidad virtual. Estos dispositivos son variados entre ellos en cuanto a los sensores, cámaras o conectividad entre otras características.\\

Por último, al ser la realidad mixta una combinación entre las dos realidades mencionadas anteriormente, la \textit{MR} puede ser utilizada a través de cascos enfocados en experiencias de realidad virtual, dispositivos enfocados para su uso en realidad aumentada o incluso desde teléfonos móviles, estos dispositivos, están empezando a ganar importancia en el sector de la realidad mixta.\\

Según un estudio realizado por Sam Barker \cite{juniperArMrmoney}, los avances en el \textit{Edge Computing}~\cite{edgeComputing}, un paradigma que permite que los servicios de computación en la nube~\cite{cloudComputing} (tecnología de computación que provee unidades computacionales de bajo coste) sean más cercanos al usuario final, y la aparición de las redes de 5G (la quinta generación de tecnología inalámbrica la cual se caracteriza por un aumento de la velocidad y menor latencia frente al 4G), acelerarán el desarrollo de la tecnología de realidad mixta en móviles.\\

Dicho autor afirma que las conexiones inalámbricas que se utilizan hoy día para procesar la información de los dispositivos deberán alcanzar velocidades mayores (usando el 5G) para una experiencia inmersiva total , además, calcula que gracias a estos avances se alcanzará un mercado de~43 billones de dólares en el año~2024, a diferencia de los 8 billones alcanzados en~2019.

\parindent=0em
\subsection{Teléfonos móviles}
\label{sec:telefonosMoviles}
\noindent

%https://www.aniwaa.com/product/vr-ar/tesseract-holoboard-enterprise-edition/

Los teléfonos móviles son un gran dispositivo ya que combinan GPS, cámara, brújula y un acelerómetro, cubriendo así las necesidades de una aplicación de realidad aumentada~\cite{arsmartphones}, además, es un dispositivo muy común entre la población lo cual facilita el desarrollo y expansión de esta realidad. Por otro lado, los teléfonos móviles se pueden utilizar también para realidad mixta utilizando unos cascos en los que se introduce el dispositivo para generar una experiencia de MR.\\

Existen distintos HMD que se utilizan para estas experiencias de realidad mixta acoplando los móviles a alguna parte de las gafas, como por ejemplo, las \textit{Holoboard Enterprise Edition} de la mano de la empresa \textit{TESSERACT} (figura~\ref{fig:mrandroidTESSERACT}). Este dispositivo es compatible a partir de Android 6.0  y como especificaciones técnicas se pueden destacar sus 82\degree~ de FOV, utiliza la tecnología SLAM para el \textit{tracking}, posee un controlador 6DoF y utiliza el \textit{tracking} por sensores IMU, además, permite experiencias colaborativas en la nube.

\begin{figure}[H]
    \centering
    \includegraphics[scale=0.3]{Images/Estado del arte/mrandroid.jpg}
    \caption[\textit{Holoboard Enterprise Edition}]{
    \textit{Holoboard Enterprise Edition\footnotemark.}
    }
    \label{fig:mrandroidTESSERACT}
\end{figure}
\footnotetext{Fuente: \url{https://www.aniwaa.com/product/vr-ar/tesseract-holoboard-enterprise-edition}}
En cambio, si estamos hablando de un dispositivo para utilizar en móviles con iOS podemos hablar del casco \textit{Bridge} (figura~\ref{fig:mriosBRIDGE}) de la empresa \textit{Occipital}. Este HMD requiere de un componente extra llamado \textit{Occipital Structure Sensor}, el cual, se utiliza para el escaneo 3D del entorno.

\begin{figure}[H]
    \centering
    \includegraphics[scale=0.3]{Images/Estado del arte/mrios.jpg}
    \caption[\textit{Occipital Bridge}]{\textit{Occipital Bridge\footnotemark.}}
    \label{fig:mriosBRIDGE}
\end{figure}

\footnotetext{Fuente: \url{https://www.aniwaa.com/product/vr-headsets/occipital-bridge/}}

Las gafas \textit{Bridge} tienen un FOV de 120º, un controlador de 6DoF y destacan por que gracias a estas se puede utilizar aplicaciones de realidad aumentada, realidad mixta y realidad virtual, también, utiliza la técnica de \textit{tracking} conocida como \textit{inside-out}.\\

Pese a que la principal diferencia entre las dos gafas es el sistema operativo para el que están destinadas, se puede observar que los precios son similares y que las gafas destinadas para iOS poseen un FOV notablemente superior.


\begin{table}[H]
\centering
\renewcommand{\arraystretch}{1.5}
\begin{tabular}{llllll}
\toprule
Dispositivo                  & Precio & DoF & FOV & \textit{SO} & \textit{Tracking} \\
\midrule
\textit{Holoboard Enterprise Edition} & \$399  & 6DoF         & 82\degree           & Android     & SLAM              \\
\textit{Occipital Bridge }            & \$349  & 6DoF         & 120\degree          & iOS         & \textit{Inside-out}\\\bottomrule       
\end{tabular}
\caption{Comparación entre ambas gafas de \textit{MR} para móviles.}
\label{cuadro:comparacionphonesMR}
\end{table}


\parindent=0em
\subsection{Hololens 2}
\label{HoloLens2Dispositivo}
\noindent

Las \textit{Microsoft HoloLens 2} son un casco desarrollado por Microsoft que salió al mercado por primera vez el día 7 de Noviembre de 2019, es el dispositivo sucesor a las Microsoft Hololens de 2016.


\begin{figure}[htbp]
\centering
    \hspace{-4mm}
    \begin{minipage}{0.5\textwidth}
        \centering
        \includegraphics[scale=0.2]{Images/Estado del arte/hololens2_2.jpeg}\\
        (a) Vista lateral.
    \end{minipage}
    \begin{minipage}{0.5\textwidth}
        \centering
        \includegraphics[scale=0.2]{Images/Estado del arte/hololens2_3.jpeg}\\
       (b) Vista superior.
    \end{minipage}\\
    \caption{Vistas de las Microsoft HoloLens 2\textsuperscript{\ref{hololens2footer}}.}
    \label{fig:vistasHoloLens2}
\end{figure}


Microsoft Hololens 2 utiliza los servicios de cloud y la IA de Microsoft, además, está diseñado con un sistema ergonómico de tal forma que no hay necesidad de quitarse las gafas ya que se coloca directamente sobre ellas (figura~\ref{fig:hololensErgonomicas}).\\
\begin{wrapfigure}{r}{.5\textwidth}
  \centering
    \includegraphics[scale=0.28]{Images/Estado del arte/hololens2_1.jpeg}
  \caption{Diseño ergonómico del dispositivo\textsuperscript{\ref{hololens2footer}}
  }
  
  \label{fig:hololensErgonomicas}
\end{wrapfigure}

Respecto a las especificaciones técnicas del casco se pueden destacar las siguientes relacionadas con los sensores: cuatro cámaras de luz visible para el seguimiento de la cabeza, dos cámaras de infrarrojos para el movimiento de los ojos y un sensor de profundidad de 1 mega píxel, además de contar con acelerómetro, giroscopio y magnetómetro.\\



El dispositivo cuenta también con seguimiento de mano, seguimiento de ojos y controles por voz.
Por último, las Microsoft HoloLens 2 tienen un peso de 566 gramos y utilizan un sistema operativo holográfico de Windows.

Actualmente están disponibles para empresas y desarrolladores, su precio depende de la elección entre distintos paquetes:

\begin{itemize}
    \item \textbf{Solo el dispositivo:} 3,500 dólares.
    \item \textbf{Dispositivo con asistencia remota:} El coste del dispositivo y 500 dólares al mes, incluye asistencia remota los 365 días.
    \item \textbf{Paquete de desarrollador:} Coste del dispositivo añadido a 99 dólares por mes, este paquete incluye Unity Pro y un crédito de 500 dólares para Azure.
\end{itemize}

\footnotetext[1]{\label{hololens2footer}{Imágenes y especificaciones obtenidas de \url{https://www.microsoft.com/es-es/hololens/hardware}.}}




\parindent=0em
\subsection{HMD Odyssey+}
\label{sec:odyssey}
\noindent

%https://www.samsung.com/hk_en/news/product/reality-headset-hmd-odyssey-plus/

El casco \textit{HMD Odyssey+} (figura~\ref{fig:hdmOdysseyVista}) es un dispositivo desarrollado por la empresa Samsung que salió al mercado en 2018, no es un casco independiente ya que requiere estar conectado a un ordenador compatible con la plataforma \textit{Windows Mixed Reality} para funcionar, se conecta al ordenador mediante un cable HDMI~2.0 y un USB~3.0.

\begin{figure}[h]
    \centering
    \includegraphics[scale=0.6]{Images/Estado del arte/samsungOdysseyplus.jpg}
    \caption[\textit{HMD Odyssey+ dispositivo al completo}]{\textit{HMD Odyssey+ dispositivo al completo}\footnotemark.}
    \label{fig:hdmOdysseyVista}
\end{figure}

\footnotetext{Fuente: \url{https://www.samsung.com/hk_en/news/product/reality-headset-hmd-odyssey-plus/}}
En cuanto a la parte de monitorización de la imagen, este casco tiene una resolución de 1.440 x 1.600 píxeles por ojo, un total de 2.880 x 1.600 píxeles combinando los dos ojos, además, el \textit{Field of View} abarca un total de 110\degree . Este dispositivo no posee tecnología de \textit{hand tracking} (por lo que controla el movimiento de manos con dos mandos que funcionan con pilas) y tampoco tiene \textit{eye tracking}, en cambio, destaca por tener dos sensores 6DoF frente al único que tiene, por ejemplo, las \textit{Hololens~2} (sección~\ref{HoloLens2Dispositivo}).\\

Por otra parte, está dotado de sensores como acelerómetro, giroscopio, brújula y sensor de proximidad (este último se utiliza para saber en qué momento se lleva puesto el casco), del mismo modo, posee una diadema que se puede  modificar para ajustar el casco a la cabeza de cada individuo, sin embargo, este dispositivo no está diseñado para ser utilizado con gafas.\\ 

Por último, tiene un sensor que ajusta automáticamente la IPD, conexión \textit{bluetooth} y un peso total de 644g teniendo en cuenta solo el HMD y un añadido de 176g si se cuenta el peso del cable.

\parindent=0em
\subsection{Magic Leap One}
\noindent

El dispositivo \textit{Magic Leap One} (figura~\ref{magicLeaponeFotterSpecs}) fue lanzado al mercado en el año 2018 de manos de la compañía Magic Leap. Principalmente este casco destaca por su ligero peso frente a otros dispositivos de 316 gramos, por otra parte, son un \textit{HMD} independiente ya que vienen con un sistema (procesador y tarjeta gráfica) para no tener que ser usados mediante conexión con un ordenador. Aunque no es necesario conectarlo al ordenador, el dispositivo tiene conexión \textit{bluetooth}, \textit{WiFi} y \textit{USB} para poder conectarse al \textit{PC} y manejar archivos o instalar aplicaciones.\\

Por otra parte, pese a que dispone de tecnología de \textit{hand tracking}, viene con un mando para simular una de las manos del usuario, asimismo, utiliza la tecnología de \textit{eye tracking} para poder interactuar con el entorno a través de la mirada. Del mismo modo, el dispositivo captura \textit{6DoF}, tiene un campo de visión de 50º y una resolución de 1.280x960 píxeles por ojo (un total de (2.560x960 píxeles en combinación con los dos ojos).
 

\begin{figure}[htbp]
\centering
    \hspace{-4mm}
    \begin{minipage}{0.5\textwidth}
        \centering
        \includegraphics[scale=0.8]{Images/Estado del arte/magicleapone1.jpg}\\
        (a) Vista del casco completo
    \end{minipage}
    \begin{minipage}{0.5\textwidth}
        \centering
        \includegraphics[scale=0.3]{Images/Estado del arte/magicleapone2.jpg}\\
       (b) Vista del mando
    \end{minipage}\\
    \caption{Dispositivo completo de las Magic Leap One\textsuperscript{\ref{magicLeaponeFotterImages}}.}
    \label{fig:vistasMagicLeapnOne}
\end{figure}

Por último, la batería recargable de litio de las \textit{Magic Leap One} permite un uso continuado de 3 horas y, además, tiene un diseño ergonómico mediante el cual se puede utilizar el casco con gafas, ya que hay un compartimento enfrente de los ojos adaptado para el hueco de unas gafas.

%https://www.businessinsider.com/magic-leap-one-creator-edition-price-specifications-battery-life-release-date-2018-8?IR=T


\footnotetext[1]{
\label{magicLeaponeFotterSpecs}{Especificaciones obtenidas de \url{https://www.businessinsider.com/magic-leap-one-creator-edition-price-specifications-battery-life-release-date-2018-8?IR=T}.}}

\footnotetext[1]{
\label{magicLeaponeFotterImages}{Imágenes sacadas de \url{https://www.estiloextra.net/magic-leap-one-las-nuevas-y-prometedoras-gafas-de-realidad-aumentada/}.}}

\parindent=0em
\subsection{DAQRI Smart Helmet}
\noindent

%https://www.linkedin.com/pulse/daqri-smart-helmet-closer-look-nathan-gaydhani

El \textit{DAQRI Smart Helmet} es un \textit{HMD} enfocado al uso industrial, está formado por un dispositivo de realidad mixta sin cables integrado en un casco duro (cascos utilizados por ejemplo en la construcción). Ya que está pensado para un uso industrial está dotado de 4 cámaras para poder capturar todo el entorno, es decir, abarcar 360º.\\

Con el objetivo de proteger al usuario en entornos industriales peligrosos, el dispositivo cuenta con una cámara termográfica (para poder controlar lugares potencialmente peligrosos debido a su temperatura) y barómetro para poder medir la presión del entorno.\\

Para el seguimiento, el \textit{DAQRI Smart Helmet} utiliza la tecnología \textit{SLAM} y posee \textit{6DoF} para el movimiento y los giros del usuario. Del mismo modo, ya que el objetivo final es proteger al usuario, el dispositivo cuenta con reconocimiento de órdenes por voz y de control de movimientos de la cabeza para manejar el \textit{HMD}, de esta forma se pueden tener las manos libres.\\

\begin{figure}[h]
    \centering
    \includegraphics[scale=0.12]{Images/Estado del arte/daqrihelmet.jpg}
    \caption{Vista del DAQRI Smart Helmet}
    \label{fig:vistaDAQRIHelmet}
\end{figure}

Finalmente, cabe recalcar que el casco goza de procesadores y un sistema operativo \textit{Android} para ser totalmente independiente, también, utiliza unas baterías intercambiables de 5700 miliamperios por hora y dispone de conectividad \textit{WiFi} para poder comunicarse por vídeo en tiempo real remotamente con otros usuarios. El casco tiene un peso total de 1 kilo y~500 gramos y su precio actual es de~15.000 dólares.

%\footnotetext{\label{daqriImagefooter}{Imagenes obtenida de: %\url{https://www.stereoscape.com/blog/2017/04/25/daqri-smart-helmet-so-m%uch-more-than-a-helmet/}.}}




\parindent=0em
\subsection{HP VR1000-127il}
\noindent

Este \textit{HMD} de la empresa \textit{HP} salió al mercado en el año 2018, es un dispositivo dependiente de un ordenador compatible con la plataforma \textit{Windows Mixed Reality}. Este casco se conecta a dicho ordenador a través de una conexión 2 en 1 que combina un \textit{HDMI} 2.0 y \textit{USB} 3.0. \\

Las gafas \textit{HP VR1000-123il} (figura~\ref{fig:hpvr1000}) utilizan dos controladores inalámbricos (que se conectan mediante \textit{bluetooth}) para controlar el movimiento de las manos, estos controladores utilizan 2 pilas AA cada uno. El dispositivo tiene un campo de visión de 90º y una resolución de 1.440×1.440 píxeles por ojo (un total de 2.880x1.440 píxeles con ambos ojos), por otro lado, el casco cuenta con \textit{6DoF} y la \textit{IPD} no se puede regular. 

\begin{figure}[H]
    \centering
    \includegraphics[scale=0.45]{Images/Estado del arte/HP VR1000.png}
    \caption{HP VR1000-123il y sus controladores.}
    \label{fig:hpvr1000}
\end{figure}

Finalmente, este \textit{HMD} no tiene un diseño centrado en su uso con gafas y tiene un peso de 834 gramos. 
\parindent=0em
\subsection{Asus HC102}
\noindent

Este dispositivo (figura~\ref{fig:asushc102}) apareció en el año 2018 de manos de la empresa \textit{Asus}, al igual que las \textit{HMD Odyssey+} (punto~\ref{sec:odyssey}), es un dispositivo que requiere ser utilizado junto a un ordenador compatible con \textit{Windows Mixed Reality}, está dotado de sensores para calcular la orientación del usuario como giroscopio, acelerómetro, magnetómetro y sensor de proximidad.\\

\begin{figure}[H]
    \centering
    \includegraphics[scale=0.45]{Images/Estado del arte/AsusHC102.jpg}
    \caption{Asus HC102 al completo.}
    \label{fig:asushc102}
\end{figure}

Este casco goza de un \textit{FOV} de 105º, una resolución de 1.440x1.440 por ojo (2.880x1440 en total) y cuenta con dos mandos (cada uno de ellos con un sensor \textit{6DoF}) que funcionan con 2 pilas del tipo AA, es decir, 4 pilas AA en total.\\

Finalmente, este \textit{HMD} permite regular la \textit{IPD} entre 55mm y 71mm, tiene dos cámaras para hacer un \textit{tracking} de tipo \textit{inside-out} y un peso de 399 gramos.
\parindent=0em
\subsection{Acer AH101-D8EY}
\noindent

Este dispositivo de la marca Acer (figura~\ref{fig:acerAH101}) fue lanzado al mercado en 2017, para utilizarlo es necesario un ordenador compatible con \textit{Windows Mixed Reality}, a diferencia de los otros HMD que se utilizan con la plataforma WMR, este dispositivo se conecta al ordenador a través de conexión \textit{bluetooth}.\\

\begin{figure}[H]
\centering
    \hspace{-4mm}
    \begin{minipage}[t]{0.5\textwidth}
        \centering
        \includegraphics[scale=0.30]{Images/Estado del arte/acerh101.png}\\
        (a) Vista del casco completo
    \end{minipage}
    \begin{minipage}[t]{0.5\textwidth}
        \centering
        \includegraphics[scale=0.42]{Images/Estado del arte/acerh101controller.png}\\
       (b) Vista de los controladores
    \end{minipage}\\
    \caption[Dispositivo \textit{AcerAH101-D8EY}]{Dispositivo \textit{AcerAH101-D8EY}\footnotemark.}
    \label{fig:acerAH101}
\end{figure}

\footnotetext{Fuente: \href{https://www.acer.com/ac/en/US/content/model/VD.R05AP.002}{\nolinkurl{https://www.acer.com/model/VD.R05AP.002}}}

El campo de visión del casco es de 100\degree~  y respecto a la resolución, es capaz de mostrar 2.880x1.440 píxeles combinando los dos ojos (1.440x1.440 píxeles en cada ojo), además, la distancia interpupilar no es ajustable y viene configurada por defecto a 62mm.\\



Por otro lado, ya que se conecta mediante \textit{bluetooth} necesita una fuente de energía de 2 pilas tipo AA, del mismo modo, el dispositivo viene con dos controladores que igualmente funcionan con 2 pilas del tipo AA cada uno y tienen 6 grados de libertad. \\

Los sensores que tiene este casco son: acelerómetro, giroscopio, magnetómetro y sensor de proximidad, además, el peso del dispositivo es de 848g. 
\parindent=0em
\subsection{Lenovo Explorer}
\noindent

La marca Lenovo puso a la venta el dispositivo \textit{Lenovo Explorer} (figura~\ref{fig:lenovoExplorer}) en el año 2017, es un casco de la realidad mixta que se utiliza a través de \textit{Windows Mixed Reality}, es decir, depende de un ordenador para ser utilizado. Este HMD se conecta a través de un cable en Y con conexión HDMI y USB~3.0 y goza de conexión \textit{bluetooth}.\\

Este dispositivo está dotado de sensor de proximidad, giroscopio, acelerómetro y magnetómetro, además, posee dos cámaras para realizar el \textit{tracking} \textit{inside-out}.\\

Por otra parte, tiene un FOV de 110\degree  y una resolución con ambos ojos de 2.880x1.440 píxeles (1.440x1.440 con cada ojo), tiene control de 6 grados de libertad  y su IPD no se puede ajustar viniendo fijo con un valor de 62mm. Este dispositivo no tiene tecnología \textit{hand tracking} ni \textit{eye tracking}, para sustituir el seguimiento de manos se utilizan dos controladores que funcionan con pilas del tipo AA. Su peso es de 380 gramos. 

%https://www.amazon.com/-/es/Lenovo-G0A20001WW-Explorer-Mixed-Reality-Auriculares/dp/B0764GKZ15


\begin{figure}[H]
    \centering
    \includegraphics[scale=0.2]{Images/Estado del arte/lenovoexplorer.jpg}
    \caption[Lenovo Explore]{\textit{Lenovo Explore}r\footnotemark.}
    \label{fig:lenovoExplorer}
\end{figure}

\footnotetext{Fuente: \href{https://www.lenovo.com/es/es/smart-devices/virtual-reality/lenovo-explorer/Lenovo-Explorer/p/G10NREAG0A2}{\nolinkurl{https://www.lenovo.com/Lenovo-Explorer/p/G10NREAG0A2}}}

\parindent=0em
\subsection{HP Reverb}
\noindent
\parindent=0em
\subsection{Comparación de dispositivos}
\noindent

