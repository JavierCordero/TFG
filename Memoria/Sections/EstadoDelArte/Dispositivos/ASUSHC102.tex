\parindent=0em
\subsection{Asus HC102}
\noindent

Este dispositivo (figura~\ref{fig:asushc102}) apareció en el año 2018 de manos de la empresa \textit{Asus}, al igual que las \textit{HMD Odyssey+} (punto~\ref{sec:odyssey}), es un dispositivo que requiere ser utilizado junto a un ordenador compatible con \textit{Windows Mixed Reality}, está dotado de sensores para calcular la orientación del usuario como giroscopio, acelerómetro, magnetómetro y sensor de proximidad.\\

\begin{figure}[H]
    \centering
    \includegraphics[scale=0.45]{Images/Estado del arte/AsusHC102.jpg}
    \caption{Asus HC102 al completo.}
    \label{fig:asushc102}
\end{figure}

Este casco goza de un \textit{FOV} de 105º, una resolución de 1.440x1.440 por ojo (2.880x1440 en total) y cuenta con dos mandos (cada uno de ellos con un sensor \textit{6DoF}) que funcionan con 2 pilas del tipo AA, es decir, 4 pilas AA en total.\\

Finalmente, este \textit{HMD} permite regular la \textit{IPD} entre 55mm y 71mm, tiene dos cámaras para hacer un \textit{tracking} de tipo \textit{inside-out} y un peso de 399 gramos.