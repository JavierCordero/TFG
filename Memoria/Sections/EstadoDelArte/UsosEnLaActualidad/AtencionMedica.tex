\parindent=0em
\subsection{Atención médica}
\noindent

%https://blogs.windows.com/windowsexperience/2018/03/08/how-mixed-reality-is-changing-the-game-for-healthcare-from-performing-live-surgeries-to-delivering-ultrasounds-in-3d/

Existen diversos usos de la realidad mixta en la atención médica, desde educación médica hasta propias aplicaciones de atención médica. En primer lugar, la empresa \textit{CAE Healthcare} es una de las líderes en simulaciones tecnológicas y en recursos para mejorar el desempeño en las clínicas. \textit{CAE VimedixAR} es un simulador de ultrasonido diseñado por dicha empresa. \\

Esta aplicación superpone hologramas de lo que se vería al realizar un ultrasonido. Estos hologramas se superponen sobre un maniquí del mundo físico, de esta forma, los profesionales pueden manipular partes anatómicas (rotándolas y escalándolas).\\

\begin{figure}[H]
    \centering
    \includegraphics[scale=0.35]{Images/Estado del arte/caehealthcareparto.jpg}
    \caption[Uso de la aplicación \textit{CAE Lucina}]{Uso de la aplicación \textit{CAE Lucina}\footnotemark.}
    \label{fig:caelucina}
\end{figure}

Otra aplicación diseñada por esta empresa es \textit{CAE Lucina} (figura \ref{fig:caelucina}). Se trata de un simulador de partos donde los profesionales pueden realizar la extracción completa del feto en cualquier situación excepcional (estando así preparados para cualquier imprevisto).\\



\footnotetext{Fuente: \href{https://blogs.windows.com/windowsexperience/2018/03/08/how-mixed-reality-is-changing-the-game-for-healthcare-from-performing-live-surgeries-to-delivering-ultrasounds-in-3d/}{\nolinkurl{https://blogs.windows.com/windowsexperience/2018/03/08/}}\label{blogWindowsFootnote}}

Por otro lado, la empresa Pearson ha desarrollado una aplicación llamada \textit{HoloPatient} la cual sirve para formar estudiantes de enfermería. El programa se basa en una experiencia que permite aprender sin necesidad de actores o pacientes que se necesitarían sin usar esta tecnología. \textit{HoloPatient} salió al mercado en abril del año 2018.\\

Existen otros proyectos como el que viene de la mano de SphereGen en colaboración con una universidad, bautizado con el nombre de \textit{Learning Heart} (figura \ref{fig:learningheartspheregen}). Esta aplicación permite a sus usuarios aprender sobre el corazón (de manera individual o colaborativa) viéndolo desde cualquier ángulo y posición. \\

\begin{figure}[h]
    \centering
    \includegraphics[scale=0.65]{Images/Estado del arte/learningheartspheregen.jpeg}
    \caption[\textit{Learning Heart} en ejecución.]{\textit{Learning Heart} en ejecución\footnotemark.}
    \label{fig:learningheartspheregen}
\end{figure}

\footnotetext{Fuente: \href{https://www.microsoft.com/en-us/p/learning-heart/9pghzvfwxpmb?activetab=pivot:overviewtab}{\nolinkurl{https://www.microsoft.com/en-us/p/learning-heart}}}

%Foto del learning 
%https://www.microsoft.com/en-us/p/learning-heart/9pghzvfwxpmb?activetab=pivot:overviewtab}


En cuanto a la radiología, \textit{DICOM Director} aporta una gran ayuda a estos expertos permitiéndoles ver y analizar todo tipo de escaneos en 3D. Los radiólogos y doctores pueden ver estas radiografías haciendo uso de sus cascos de realidad mixta de Windows, igualmente, la aplicación genera un modelo 3D de esas imágenes (figura \ref{fig:dicomDirector}) los cuales se pueden rotar y analizar. De esta manera, \textit{DICOM Director} se utiliza también como herramienta de comunicación entre doctores para compartir estos modelos.

\begin{figure}[H]
    \centering    \includegraphics[scale=0.2]{Images/Estado del arte/dicomdirector.jpg}
    \caption{Usuario rotando el modelo 3D de una radiografía en\textit{DICOM Director}\textsuperscript{\ref{blogWindowsFootnote}}.}
    \label{fig:dicomDirector}
\end{figure}

Para finalizar, la empresa \textit{Visual 3D Medical Science and Technology Development CO. LLC} ha desarrollado un producto que ha sido usado en China para realizar 200 operaciones (de rodilla, cadera y columna vertebral) gracias a la realidad mixta. Incluso se está utilizando para preparar a los cirujanos antes de entrar a la sala de operaciones, reproduciendo la operación a través de las gafas. 






