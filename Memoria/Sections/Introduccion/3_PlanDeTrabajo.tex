\parindent=0em
\section{Plan de trabajo}
\noindent

Se puede dividir la planificación en dos grandes bloques: efecto de oclusión en teléfonos móviles y búsqueda de rutas.\\

En primer lugar, en esta primera parte, se deberá investigar sobre el funcionamiento de ARCore a la hora de obtener información sobre la profundidad de los objetos del mundo físico en tiempo real utilizando una única cámara monocular.\\

Después de conocer dicha información, se realizará una investigación leyendo libros y artículos sobre realidad mixta para conocer mas información sobre la oclusión. Es importante que estos conocimientos estén enfocados a teléfonos móviles y no utilicen cámaras distintas a una única cámara monocular, como se ha mencionado anteriormente.\\

Para poner a prueba los conocimientos obtenidos anteriormente, en esta parte se implementarán 3 aplicaciones:

\begin{itemize}
    \item \textbf{Aplicación del \textit{shader} de profundidad:} Se desarrollará una aplicación para crear un mapa de profundidad (sección~\ref{sec:oclusion}).
    
    \item \textbf{Aplicación del \textit{shader} de oclusión:} Aplicación para probar la efectividad y precisión del \textit{shader} de oclusión (sección~\ref{shaderOclusionSec}).
    
    \item \textbf{Aplicación de la nube de puntos:} Esta aplicación se utilizará para comprobar la efectividad del \textit{shader} de oclusión a través de una nube de puntos (sección~\ref{sec:nubeDePuntos}).
\end{itemize}

Con estas tres aplicaciones se comprobará la precisión a la hora de generar un efecto de oclusión de distintas formas para quedarse con la más precisa y más eficiente.\\

En la segunda parte, se averiguará qué funciones aporta el sistema de búsqueda Mapbox con Unity. Una vez obtenida dicha información, se tomará como objetivo desarrollar una aplicación para probar esas funcionalidades.\\

Esta aplicación se implementará para obtener unas coordenadas GPS y transformar esa latitud y longitud en coordenadas de Unity. Una vez hecho esto, se transformará la posición obtenida en Unity a una posición en el entorno de ARCore. Todo esto se hará con el objetivo final de poder colocar un objeto virtual dadas unas coordenadas de latitud y longitud con exactitud.









