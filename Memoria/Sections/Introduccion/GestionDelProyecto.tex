\parindent=0em
\section{Gestión del proyecto}
\noindent

Se utilizarán distintas herramientas para desarrollar y gestionar el proyecto. En primer lugar, respecto al desarrollo de la memoria se utilizará \textbf{Overleaf} debido a su funcionalidad de trabajo colaborativo (al ser únicamente dos personas se puede utilizar sin comprar ningún plan) y la posibilidad de guardar los archivos en la nube así como de descargarlos.\\

Para el control de versiones se utilizará git. Todas las aplicaciones y proyectos se subirán a un repositorio de la plataforma \textbf{Github}. Con el objetivo de separar la dos grandes partes de este Trabajo de Fin de Grado, la oclusión y los mapas, se creará una rama ``MAPS'' donde se trabajará con los proyectos relacionados con los distintos proveedores de búsquedas de rutas. Por otro lado, se utilizará la rama ``master'' como rama principal para la parte de la oclusión.\\

Por otro lado, en el ámbito de la implementación de código, se utilizará el IDE (del inglés \textit{Integrated Development Environment}) \textbf{Visual Studio Community 2020}. Este es el IDE elegido debido a la experiencia que hemos obtenido en él durante los distintos cursos del grado y la facilidad que aporta para trabajar con Unity a través de instalar algunos paquetes.\\

Para desarrollar las aplicaciones se utilizará uno de los motores de videojuegos punteros, Unity. Este motor ha sido escogido gracias a la posibilidad que ofrece de integrar ARCore así como su facilidad para integrar distintos proveedores de mapas. La versión que se utilizará será \textbf{Unity 2019.3.1f1}. \\

Con el objetivo de generar una experiencia de realidad mixta se utilizará la tecnología de Google \textbf{ARCore}. A la hora de utilizar ARCore con Unity, si en el proyecto se va a incluir Mapbox, se utilizará la versión de ARCore que viene incluida con Mapbox. En el caso de utilizar ARCore sin incluir Mapbox, se utilizará la versión \textbf{ARCore 3.1.3} ya que en esta versión se incluye la actualización donde se permite utilizar la información de la profundidad.