\parindent=0em
\section{Project Managment}
\noindent

Different tools will be used to develop and manage the proyect. 
Se utilizarán distintas herramientas para desarrollar y gestionar el proyecto. Firstly, regarding the memory development, \textbf{Overleaf} will be used due to its collaborative work functionality (the free plan is available for two people) and the possibility of saving files in the cloud as well as to download them.\\

For version control git will be used. All applications and projects will be uploaded to a repository on the \textbf{Github} platform. In order to separate the two main parts of this Final Degree Project, the occlusion and the maps parts, a ``MAPS'' branch will be created where we will work with projects related to the different route search providers. On the other hand, the ``master'' branch will be used as the main branch for the occlusion part.\\

On the other hand, in the field of code implementation, the IDE (\textit{Integrated Development Environment}) \textbf{Visual Studio Community 2020} will be used. This is the IDE chosen due to the experience we have obtained in it during the different grade courses and the ease it brings to work with Unity through installing some packages.\\

One of the leading video game engines, Unity, will be used to develop the applications. This engine has been chosen due to the possibility that it offers to integrate ARCore as well as its facility to integrate different map providers. The version to be used will be \textbf {Unity 2019.3.1f1}.\\


With the main goal of generate a mixed reality experience, Google technology \textbf {ARCore} will be used. When using ARCore with Unity, if Mapbox will be included in the project, the version of ARCore that is included with Mapbox will be used. In the case of using ARCore without including Mapbox, the version \textbf {ARCore 3.1.3} will be used since this version includes the update where the depth information is brought to the developers.
