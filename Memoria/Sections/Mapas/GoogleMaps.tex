\parindent=0em
\section{Google Maps}
\noindent

La empresa Google ha creado una plataforma llamada \textit{Google Maps Platform}~\cite{googlemapsPlatform}, esta plataforma ofrece 3 servicios principales:

\begin{itemize}
    \item \textit{\textbf{Maps}}: Permite crear experiencias simples y personalizadas de mapas estáticos y dinámicos, imágenes de \textit{Street View} y vistas en 360°.
    
    \item \textit{\textbf{Routes}}: Este servicio permite a los usuarios conocer la mejor forma de ir de un punto a otro a través de rutas, ya sea en transporte público, en bici, en coche o a pie, además, calcula la duración del trayecto y su distancia.
    
    \item \textit{\textbf{Places}}: Permite a los usuarios conocer información sobre más de 100 millones de ubicaciones, asimismo, permite conocer la ubicación del dispositivo sin utilizar el GPS (a través de las torres de telefonía y nodos WiFi).
    
\end{itemize}

Para poder hacer uso de esta plataforma, es necesario añadir una cuenta de financiación ya que todos sus planes son de pago. \textit{Google Maps Platform} se puede utilizar en Android, iOS e incluso en un entorno web.