\parindent=0em
\section{HERE}
\noindent
%https://developer.here.com/products/here-sdk

HERE es una aplicación de la mano de \textit{HERE Technologies}, esta aplicación posee distintos SDK para desarrolladores. Pese a ser una aplicación gratuita, existen distintos planes de pago que añaden funcionalidades extra. Las funcionalidades que aporta HERE son las siguientes:

\begin{itemize}
    \item \textbf{Instrucciones en tiempo real:} Aporta señales visuales y auditivas para la navegación en tiempo real, además, transmite información sobre señalización y carriles.
    
    \item \textbf{Modo \textit{Offline}}: Permite a los usuarios la búsqueda de lugares así como la navegación sin necesidad de conexión a internet.
    
    \item \textbf{Gran disponibilidad:} HERE está disponible en 190 países y 60 idiomas.
    
    \item \textbf{Visualización del terreno:} Esta funcionalidad añade distintos modos de visualización del entorno: peatón, tráfico, terreno...
    
    \item \textbf{Plantillas de interfaz de usuario:} Añade una forma rápida de modificar colores, tamaños y cambios en la lógica a través de plantillas.
    
    \item \textbf{Personalización del mapa:} Personalización del mapa desde cambios de color de los iconos hasta destacar elementos importantes. Además, permite editar propiedades de los elementos como carreteras o edificios.
    
    \item \textbf{Obtención de información:} Gracias a esta funcionalidad se puede obtener información detallada de las carreteras como señales, límites de velocidad...
\end{itemize}

Esta aplicación se puede utilizar en móviles (Android e iOS), a través de Javascript o con una REST API.


