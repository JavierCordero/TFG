\parindent=0em
\section{Introducción}
\noindent

Existen distintos sistemas de búsqueda de rutas actualmente donde cada cual aporta funcionalidades variadas. Estas tecnologías están relacionadas con la búsqueda del camino más corto entre dos puntos, dicha búsqueda está centrada mayoritariamente en el \textbf{Algoritmo de Dijsktra}.\\

Principalmente estas tecnologías se basan en la utilización del GPS, del inglés \textit{Global Positioning System} para conocer la ubicación del usuario. El GPS es un servicio que proporciona a los usuarios información sobre posicionamiento, navegación y cronometría~\cite{GPSGOV}. Este sistema está constituido por tres segmentos: el segmento espacial, el segmento de control y el segmento del usuario.\\

La ubicación del GPS se obtiene mediante un proceso lento debido a que se calcula mediante la recepción de datos de unos cuantos satélites los cuales dan vueltas a la Tierra. Es por ello que apareció el conocido como \textbf{A-GPS} del inglés \textit{Assisted GPS}.\\

La información de ubicación obtenida gracias al A-GPS\footnotemark~ es más rápida ya que se envía una señal a un servidor externo con la identificación de la antena obteniendo como respuesta los satélites ubicados encima del usuario. El inconveniente del A-GPS es la necesidad de poseer una tarifa de datos, aun así, aumenta considerablemente la velocidad a la hora de iniciar una navegación.

\footnotetext{Fuente: \href{https://www.xataka.com/moviles/que-es-el-a-gps}{\nolinkurl{https://www.xataka.com/moviles/que-es-el-a-gps}}}



