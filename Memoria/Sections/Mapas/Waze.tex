\parindent=0em
\section{Waze}
\noindent

%https://developers.google.com/waze

Waze es una aplicación desarrollada por Waze Mobile que además de poder ser utilizada por los usuarios finales, aporta funcionalidades a los desarrolladores para crear sistemas de búsqueda de rutas.\\

Aunque Waze ofrece distintos SDK, en este caso el relevante es \textbf{\textit{Waze Transport SDK}}. Este SDK de pago ofrece lo siguiente:

\begin{itemize}
    \item \textbf{Obtención del progreso de la ruta y puntos de la ruta:} Esta funcionalidad añade la posibilidad de conocer el tiempo estimado de llegada a un punto.
    
    \item \textbf{Navegación:} Aporta a los conductores la funcionalidad de calcular rutas de forma rápida evitando incidencias en carreteras.

    \item \textbf{Obtención de información:} Esta función del SDK permite obtener la información de las rutas.
\end{itemize}

Waze está enfocado a ser utilizado únicamente en dispositivos Android e iOS.