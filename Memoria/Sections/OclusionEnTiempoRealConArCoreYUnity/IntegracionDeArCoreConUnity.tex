\parindent=0em
\section{Integración de ARCore con Unity}
\noindent

La integración de ARCore en Unity se realiaza mediante los llamadaos packages (paquetes) de Unity. 
Dependiendo del tipo de efecto que queramos generar dentro de nuestra aplicación tendremos que dar uso de uno u otro por las diversas características que presentan.

Evidentemente tenemos que descargar primero la SDK de ArCore de para Unity.

Para nuestro caso, necesitamos la nube de puntos proporcionada por ARCore y explicada en el siguiente apartado, por lo que descargaremos e importaremos los paquetes ArCore, XR Legacy Input y multiplayer HLAPI. Dentro de las opciones de Build de Unity, establecemos la plataforma por defecto la de Android y añadimos la escena que queramos dentro de nuestra aplicación. En PlayerSettings, tenemos que quitar el tipo de renderizado Vulkan y añadir .NET 4X, así como establecer el mínimo Api Level a 24. Finalmente, habilitar ARCore supported dentro de PlayerSettings -> XRSettings.
