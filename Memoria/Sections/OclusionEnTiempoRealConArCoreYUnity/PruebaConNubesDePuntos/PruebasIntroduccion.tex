\parindent=0em
\subsection{Introducción}
\noindent

Teóricamente las nubes de puntos deberían funcionar sin demasiada complicación debido a la sencillez de su cálculo, pero en la práctica encontramos varios fallos que impiden su correcto funcionamiento.\\

Existen varias maneras de utilizar la nube de puntos para distintos fines. Para el propósito de este estudio, se hace uso de esta para generar la oclusión de los objetos virtuales con los que se encuentran en el mundo real utilizando el \textit{shader de oclusión}. \\

En este caso, se dará uso de las posiciones generadas por ARCore para crear objetos propios para la oclusión. Este será el resultado de analizar tanto el rendimiento de la nube de puntos (tanto en tiempo como en consumo de recursos) como el porcentaje de acierto de la oclusión resultante.\\