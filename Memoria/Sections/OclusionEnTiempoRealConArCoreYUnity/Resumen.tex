\parindent=0em
\section*{Resumen}
\noindent

En este capítulo se ha tratado la integración de ARCore en Unity. Esta integración es necesaria para las aplicaciones de prueba que se han desarrollado utilizando la nube de puntos de ARCore. Del mismo modo, se ha explicado el funcionamiento a fondo de dicha nube de puntos para poder ser utilizada en las aplicaciones de prueba.\\

Por otra parte, una vez que se ha comprendido el funcionamiento de la nube de puntos, se pueden distinguir dos tipos de \textit{shaders}, el de profundidad y el de oclusión. Ambos \textit{shaders} buscan utilizar la nube de puntos de algún modo para generar un efecto de oclusión.\\

Por último, se han estudiado distintas pruebas con dicha nube de cara a medir la eficiencia y la precisión del efecto de oclusión generado con uno de los \textit{shaders}.\\

En el próximo capítulo se profundizará en el funcionamiento de distintas tecnologías de sistema de búsqueda de rutas y de cómo se pueden relacionar con ARCore. Este capítulo se ha visto afectado en contenido por la pandemia del coronavirus o COVID-19. Esta pandemia ha disminuido considerablemente el desarrollo de aplicaciones y pruebas relacionadas con la búsqueda de rutas.